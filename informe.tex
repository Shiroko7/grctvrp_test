%% bare_conf.tex
%% V1.4
%% 2012/12/27
%% by Michael Shell
%% See:
%% http://www.michaelshell.org/
%% for current contact information.
%%
%% This is a skeleton file demonstrating the use of IEEEtran.cls
%% (requires IEEEtran.cls version 1.8 or later) with an IEEE conference paper.
%%
%% Support sites:
%% http://www.michaelshell.org/tex/ieeetran/
%% http://www.ctan.org/tex-archive/macros/latex/contrib/IEEEtran/
%% and
%% http://www.ieee.org/

%%*************************************************************************
%% Legal Notice:
%% This code is offered as-is without any warranty either expressed or
%% implied; without even the implied warranty of MERCHANTABILITY or
%% FITNESS FOR A PARTICULAR PURPOSE! 
%% User assumes all risk.
%% In no event shall IEEE or any contributor to this code be liable for
%% any damages or losses, including, but not limited to, incidental,
%% consequential, or any other damages, resulting from the use or misuse
%% of any information contained here.
%%
%% All comments are the opinions of their respective authors and are not
%% necessarily endorsed by the IEEE.
%%
%% This work is distributed under the LaTeX Project Public License (LPPL)
%% ( http://www.latex-project.org/ ) version 1.3, and may be freely used,
%% distributed and modified. A copy of the LPPL, version 1.3, is included
%% in the base LaTeX documentation of all distributions of LaTeX released
%% 2003/12/01 or later.
%% Retain all contribution notices and credits.
%% ** Modified files should be clearly indicated as such, including  **
%% ** renaming them and changing author support contact information. **
%%
%% File list of work: IEEEtran.cls, IEEEtran_HOWTO.pdf, bare_adv.tex,
%%                    bare_conf.tex, bare_jrnl.tex, bare_jrnl_compsoc.tex,
%%                    bare_jrnl_transmag.tex
%%*************************************************************************

% *** Authors should verify (and, if needed, correct) their LaTeX system  ***
% *** with the testflow diagnostic prior to trusting their LaTeX platform ***
% *** with production work. IEEE's font choices can trigger bugs that do  ***
% *** not appear when using other class files.                            ***
% The testflow support page is at:
% http://www.michaelshell.org/tex/testflow/



% Note that the a4paper option is mainly intended so that authors in
% countries using A4 can easily print to A4 and see how their papers will
% look in print - the typesetting of the document will not typically be
% affected with changes in paper size (but the bottom and side margins will).
% Use the testflow package mentioned above to verify correct handling of
% both paper sizes by the user's LaTeX system.
%
% Also note that the "draftcls" or "draftclsnofoot", not "draft", option
% should be used if it is desired that the figures are to be displayed in
% draft mode.
%
\documentclass[conference]{IEEEtran}
% Add the compsoc option for Computer Society conferences.
%
% If IEEEtran.cls has not been installed into the LaTeX system files,
% manually specify the path to it like:
% \documentclass[conference]{../sty/IEEEtran}





% Some very useful LaTeX packages include:
% (uncomment the ones you want to load)


% *** MISC UTILITY PACKAGES ***
%
%\usepackage{ifpdf}
% Heiko Oberdiek's ifpdf.sty is very useful if you need conditional
% compilation based on whether the output is pdf or dvi.
% usage:
% \ifpdf
%   % pdf code
% \else
%   % dvi code
% \fi
% The latest version of ifpdf.sty can be obtained from:
% http://www.ctan.org/tex-archive/macros/latex/contrib/oberdiek/
% Also, note that IEEEtran.cls V1.7 and later provides a builtin
% \ifCLASSINFOpdf conditional that works the same way.
% When switching from latex to pdflatex and vice-versa, the compiler may
% have to be run twice to clear warning/error messages.






% *** CITATION PACKAGES ***
%
%\usepackage{cite}
% cite.sty was written by Donald Arseneau
% V1.6 and later of IEEEtran pre-defines the format of the cite.sty package
% \cite{} output to follow that of IEEE. Loading the cite package will
% result in citation numbers being automatically sorted and properly
% "compressed/ranged". e.g., [1], [9], [2], [7], [5], [6] without using
% cite.sty will become [1], [2], [5]--[7], [9] using cite.sty. cite.sty's
% \cite will automatically add leading space, if needed. Use cite.sty's
% noadjust option (cite.sty V3.8 and later) if you want to turn this off
% such as if a citation ever needs to be enclosed in parenthesis.
% cite.sty is already installed on most LaTeX systems. Be sure and use
% version 4.0 (2003-05-27) and later if using hyperref.sty. cite.sty does
% not currently provide for hyperlinked citations.
% The latest version can be obtained at:
% http://www.ctan.org/tex-archive/macros/latex/contrib/cite/
% The documentation is contained in the cite.sty file itself.






% *** GRAPHICS RELATED PACKAGES ***
%
\ifCLASSINFOpdf
  % \usepackage[pdftex]{graphicx}
  % declare the path(s) where your graphic files are
  % \graphicspath{{../pdf/}{../jpeg/}}
  % and their extensions so you won't have to specify these with
  % every instance of \includegraphics
  % \DeclareGraphicsExtensions{.pdf,.jpeg,.png}
\else
  % or other class option (dvipsone, dvipdf, if not using dvips). graphicx
  % will default to the driver specified in the system graphics.cfg if no
  % driver is specified.
  % \usepackage[dvips]{graphicx}
  % declare the path(s) where your graphic files are
  % \graphicspath{{../eps/}}
  % and their extensions so you won't have to specify these with
  % every instance of \includegraphics
  % \DeclareGraphicsExtensions{.eps}
\fi
% graphicx was written by David Carlisle and Sebastian Rahtz. It is
% required if you want graphics, photos, etc. graphicx.sty is already
% installed on most LaTeX systems. The latest version and documentation
% can be obtained at: 
% http://www.ctan.org/tex-archive/macros/latex/required/graphics/
% Another good source of documentation is "Using Imported Graphics in
% LaTeX2e" by Keith Reckdahl which can be found at:
% http://www.ctan.org/tex-archive/info/epslatex/
%
% latex, and pdflatex in dvi mode, support graphics in encapsulated
% postscript (.eps) format. pdflatex in pdf mode supports graphics
% in .pdf, .jpeg, .png and .mps (metapost) formats. Users should ensure
% that all non-photo figures use a vector format (.eps, .pdf, .mps) and
% not a bitmapped formats (.jpeg, .png). IEEE frowns on bitmapped formats
% which can result in "jaggedy"/blurry rendering of lines and letters as
% well as large increases in file sizes.
%
% You can find documentation about the pdfTeX application at:
% http://www.tug.org/applications/pdftex





% *** MATH PACKAGES ***
%
%\usepackage[cmex10]{amsmath}
% A popular package from the American Mathematical Society that provides
% many useful and powerful commands for dealing with mathematics. If using
% it, be sure to load this package with the cmex10 option to ensure that
% only type 1 fonts will utilized at all point sizes. Without this option,
% it is possible that some math symbols, particularly those within
% footnotes, will be rendered in bitmap form which will result in a
% document that can not be IEEE Xplore compliant!
%
% Also, note that the amsmath package sets \interdisplaylinepenalty to 10000
% thus preventing page breaks from occurring within multiline equations. Use:
%\interdisplaylinepenalty=2500
% after loading amsmath to restore such page breaks as IEEEtran.cls normally
% does. amsmath.sty is already installed on most LaTeX systems. The latest
% version and documentation can be obtained at:
% http://www.ctan.org/tex-archive/macros/latex/required/amslatex/math/





% *** SPECIALIZED LIST PACKAGES ***
%
%\usepackage{algorithmic}
% algorithmic.sty was written by Peter Williams and Rogerio Brito.
% This package provides an algorithmic environment fo describing algorithms.
% You can use the algorithmic environment in-text or within a figure
% environment to provide for a floating algorithm. Do NOT use the algorithm
% floating environment provided by algorithm.sty (by the same authors) or
% algorithm2e.sty (by Christophe Fiorio) as IEEE does not use dedicated
% algorithm float types and packages that provide these will not provide
% correct IEEE style captions. The latest version and documentation of
% algorithmic.sty can be obtained at:
% http://www.ctan.org/tex-archive/macros/latex/contrib/algorithms/
% There is also a support site at:
% http://algorithms.berlios.de/index.html
% Also of interest may be the (relatively newer and more customizable)
% algorithmicx.sty package by Szasz Janos:
% http://www.ctan.org/tex-archive/macros/latex/contrib/algorithmicx/




% *** ALIGNMENT PACKAGES ***
%
%\usepackage{array}
% Frank Mittelbach's and David Carlisle's array.sty patches and improves
% the standard LaTeX2e array and tabular environments to provide better
% appearance and additional user controls. As the default LaTeX2e table
% generation code is lacking to the point of almost being broken with
% respect to the quality of the end results, all users are strongly
% advised to use an enhanced (at the very least that provided by array.sty)
% set of table tools. array.sty is already installed on most systems. The
% latest version and documentation can be obtained at:
% http://www.ctan.org/tex-archive/macros/latex/required/tools/


% IEEEtran contains the IEEEeqnarray family of commands that can be used to
% generate multiline equations as well as matrices, tables, etc., of high
% quality.




% *** SUBFIGURE PACKAGES ***
%\ifCLASSOPTIONcompsoc
%  \usepackage[caption=false,font=normalsize,labelfont=sf,textfont=sf]{subfig}
%\else
%  \usepackage[caption=false,font=footnotesize]{subfig}
%\fi
% subfig.sty, written by Steven Douglas Cochran, is the modern replacement
% for subfigure.sty, the latter of which is no longer maintained and is
% incompatible with some LaTeX packages including fixltx2e. However,
% subfig.sty requires and automatically loads Axel Sommerfeldt's caption.sty
% which will override IEEEtran.cls' handling of captions and this will result
% in non-IEEE style figure/table captions. To prevent this problem, be sure
% and invoke subfig.sty's "caption=false" package option (available since
% subfig.sty version 1.3, 2005/06/28) as this is will preserve IEEEtran.cls
% handling of captions.
% Note that the Computer Society format requires a larger sans serif font
% than the serif footnote size font used in traditional IEEE formatting
% and thus the need to invoke different subfig.sty package options depending
% on whether compsoc mode has been enabled.
%
% The latest version and documentation of subfig.sty can be obtained at:
% http://www.ctan.org/tex-archive/macros/latex/contrib/subfig/




% *** FLOAT PACKAGES ***
%
%\usepackage{fixltx2e}
% fixltx2e, the successor to the earlier fix2col.sty, was written by
% Frank Mittelbach and David Carlisle. This package corrects a few problems
% in the LaTeX2e kernel, the most notable of which is that in current
% LaTeX2e releases, the ordering of single and double column floats is not
% guaranteed to be preserved. Thus, an unpatched LaTeX2e can allow a
% single column figure to be placed prior to an earlier double column
% figure. The latest version and documentation can be found at:
% http://www.ctan.org/tex-archive/macros/latex/base/


%\usepackage{stfloats}
% stfloats.sty was written by Sigitas Tolusis. This package gives LaTeX2e
% the ability to do double column floats at the bottom of the page as well
% as the top. (e.g., "\begin{figure*}[!b]" is not normally possible in
% LaTeX2e). It also provides a command:
%\fnbelowfloat
% to enable the placement of footnotes below bottom floats (the standard
% LaTeX2e kernel puts them above bottom floats). This is an invasive package
% which rewrites many portions of the LaTeX2e float routines. It may not work
% with other packages that modify the LaTeX2e float routines. The latest
% version and documentation can be obtained at:
% http://www.ctan.org/tex-archive/macros/latex/contrib/sttools/
% Do not use the stfloats baselinefloat ability as IEEE does not allow
% \baselineskip to stretch. Authors submitting work to the IEEE should note
% that IEEE rarely uses double column equations and that authors should try
% to avoid such use. Do not be tempted to use the cuted.sty or midfloat.sty
% packages (also by Sigitas Tolusis) as IEEE does not format its papers in
% such ways.
% Do not attempt to use stfloats with fixltx2e as they are incompatible.
% Instead, use Morten Hogholm'a dblfloatfix which combines the features
% of both fixltx2e and stfloats:
%
% \usepackage{dblfloatfix}
% The latest version can be found at:
% http://www.ctan.org/tex-archive/macros/latex/contrib/dblfloatfix/




% *** PDF, URL AND HYPERLINK PACKAGES ***
%
%\usepackage{url}
% url.sty was written by Donald Arseneau. It provides better support for
% handling and breaking URLs. url.sty is already installed on most LaTeX
% systems. The latest version and documentation can be obtained at:
% http://www.ctan.org/tex-archive/macros/latex/contrib/url/
% Basically, \url{my_url_here}.




% *** Do not adjust lengths that control margins, column widths, etc. ***
% *** Do not use packages that alter fonts (such as pslatex).         ***
% There should be no need to do such things with IEEEtran.cls V1.6 and later.
% (Unless specifically asked to do so by the journal or conference you plan
% to submit to, of course. )


% correct bad hyphenation here
\hyphenation{op-tical net-works semi-conduc-tor}

\usepackage[noend]{algpseudocode}

\begin{document}

%
% paper title
% can use linebreaks \\ within to get better formatting as desired
% Do not put math or special symbols in the title.
\title{Diseño de Rutas Para Recolección de Materiales Valiosos.}

% author names and affiliations
% use a multiple column layout for up to three different
% affiliations
\author{\IEEEauthorblockN{Felipe Olavarría R.}
\IEEEauthorblockA{Departamento de Informática\\
Universidad Técnica Federíco Santa María\\
Avenida España 1680, Valparaíso, Chile \\\\
Email: felipe.olavarria@sansano.usm.cl}}

% conference papers do not typically use \thanks and this command
% is locked out in conference mode. If really needed, such as for
% the acknowledgment of grants, issue a \IEEEoverridecommandlockouts
% after \documentclass

% for over three affiliations, or if they all won't fit within the width
% of the page, use this alternative format:
% 
%\author{\IEEEauthorblockN{Michael Shell\IEEEauthorrefmark{1},
%Homer Simpson\IEEEauthorrefmark{2},
%James Kirk\IEEEauthorrefmark{3}, 
%Montgomery Scott\IEEEauthorrefmark{3} and
%Eldon Tyrell\IEEEauthorrefmark{4}}
%\IEEEauthorblockA{\IEEEauthorrefmark{1}School of Electrical and Computer Engineering\\
%Georgia Institute of Technology,
%Atlanta, Georgia 30332--0250\\ Email: see http://www.michaelshell.org/contact.html}
%\IEEEauthorblockA{\IEEEauthorrefmark{2}Twentieth Century Fox, Springfield, USA\\
%Email: homer@thesimpsons.com}
%\IEEEauthorblockA{\IEEEauthorrefmark{3}Starfleet Academy, San Francisco, California 96678-2391\\
%Telephone: (800) 555--1212, Fax: (888) 555--1212}
%\IEEEauthorblockA{\IEEEauthorrefmark{4}Tyrell Inc., 123 Replicant Street, Los Angeles, California 90210--4321}}




% use for special paper notices
%\IEEEspecialpapernotice{(Invited Paper)}




% make the title area
\maketitle

% As a general rule, do not put math, special symbols or citations
% in the abstract
\begin{abstract}
Este articulo trata el caso especial del problema de diseño de rutas para materiales valiosos. Conocido como \textit{Risk-constrained Cash-in-Transit Vehicle
Routing Problem (RCTVRP)}. Se proponen dos algoritmos, uno codicioso que construye soluciones bajo una heurística que se aprovecha de las condiciones de riesgo y otro de búsqueda local usando Simulated Annealing (SA). Estas técnicas se prueba en instancias propuestas construidas especialmente para tratar esta variante.

Keywords: Vehicle routing problem, Heuristic, Metaheuristic, RCTVRP.
\end{abstract}

% no keywords




% For peer review papers, you can put extra information on the cover
% page as needed:
% \ifCLASSOPTIONpeerreview
% \begin{center} \bfseries EDICS Category: 3-BBND \end{center}
% \fi
%
% For peerreview papers, this IEEEtran command inserts a page break and
% creates the second title. It will be ignored for other modes.
\IEEEpeerreviewmaketitle



\section{Introducción}

El problema de diseño de rutas, \textit{The Vehicle Routing Problem: VRP}, fue introducido en 1959 \cite{vrp}, es uno de los problemas más estudiados en la investigación de operaciones. Es de suma importancia en industrias relacionadas con el transporte o recolección de productos o personal. 

Existe un gran número de variaciones de este problema, cada una enfocada a distintas propiedades e industrias \cite{taxonomic}. VRP pertenece a la clase de problemas NP-hard y sus principales aplicaciones residen en largas instancias. Por lo que debido a su utilidad práctica, el uso de meta-heurísticas ha sido el principal enfoque en para resolver este tipo de problemas\cite{BRAEKERS2016300}.  


Este artículo se centrara en la versión del diseño de rutas para materiales valiosos. \textit{Risk-constrained Cash-in-Transit Vehicle Routing Problem: RCVRP}, introducido el 2015 en la literatura \cite{TALARICO2015457}. La principal característica de este caso, es limitar el riesgo de cada ruta para evitar perdidas. Por lo que el nuevo modelo propone nuevas restricciones y métricas para evaluar el riesgo. 

Informalmente el problema se puede definir de la siguiente manera: Dado un punto de deposito se requiere visitar cada cliente para recolectar dinero. El objetivo determinar rutas eficientes y seguras para satisfacer la demanda de los clientes y regresar al origen. A cada ruta se le asigna un vehículo, partiendo desde el deposito este recolecta dinero de un sub-conjunto de clientes y regresa al origen. Cada arco que recorre el vehículo tiene un riesgo asociado, el cual es proporcional a la distancia y tiempo recorridos en el arco y a la cantidad de dinero transportada. El riesgo total de la ruta equivale a la suma de cada riesgo individual de cada par de nodos recorrido. Este esta limitado por un parámetro predefinido de riesgo que se define dependiendo de distintos factores tales como la cantidad de dinero, las características del sector y la aptitud de la compañía a la perdida. 

A continuación se detallará el estado del arte para problemas similares, enfocado en el uso de meta-heurísticas. Se propondrá una técnica \textit{greedy} para resolver el problema y se evaluaran los resultados con instancias utilizadas anteriormente \cite{TALARICO2015457}\cite{TALARICO2017547}\cite{RADOJICIC2018486}.


\section{Estado del Arte}

\subsection{Formulación del problema}

Anteriormente a la concepción de esta variante se concebía el riesgo bajo factores del impacto de un accidente y sus repercusiones. Esta variante se conoce como \textit{Hazard} \cite{pijawka1985risk} \cite{hazard1}, la cual se considera el transporte de materiales peligrosos que pueden dañar el medio ambiente de un lugar si ocurre un accidente.

Una variante similar busca aumentar la seguridad generando rutas que sean difíciles de predecir, esto se logra buscando soluciones de buena calidad pero distintas en su estructura. Se conoce bajo el nombre de \textit{The Peripatetic Vehicle Routing Problem} \cite{ppp2}\cite{pppp}.


Formalmente \textit{RCTVRP} se define de la siguiente manera:

Sobre un grafo dirigido, $G=(V,A)$ se consideran los nodos de partida ($s$) y llegada ($e$) en el origen, todas las rutas comienzan y terminan en estos puntos. El conjunto de nodos $V=\{s,e\} \cup N$, corresponde al conjunto de clientes $N=\{1,...,n\}$ y el deposito. Cada cliente $i \in N$ tiene una demanda no negativa $d_i$ que representa el dinero a ser recogido en el nodo $i$. La demanda de los puntos del deposito se consideran nulas ($d_s=d_e=0$). El conjunto de arcos se define como $A=(N\times N) \cup (\{s\}\times N) \cup (N\times \{e\})$. Una distancia no negativa $c_{ij}$ es asociada a cada arco $(i,j) \in A$. Todos los vehículos comienzan vacíos ($d_s=0$) en el deposito y forman una única ruta, visitando una serie de clientes antes de volver al punto de origen ($e$), donde el dinero (demanda) es depositado. Al comienzo del recorrido, cada índice de riesgo del vehículo es cero. Un vehículo viajando entre los nodos $i$ y $j$ incrementa su riesgo por un valor equivalente al producto del dinero transportado y la distancia  (o tiempo) entre los nodos $c_{ij}$ \cite{TALARICO2015457} .



\subsection{Definición del riesgo}

El riesgo de que ocurra algo se puede separar en dos factores. La probabilidad de que el evento ocurra y sus consecuencias ($R_{event}=p_{event}\cdot C_{event}$). Suponiendo que la probabilidad de que un robo ocurra en el arco $(i,j)$ es independiente que ocurra en cualquier otro, el riesgo en un arco  de la ruta $r$ puede ser definido como:

\begin{equation}
    \sum_{(i,j)\in r} p_{ij}\cdot v_{ij} \cdot D^r_i
\end{equation}

donde $p$ es la probabilidad que ocurra el robo, $v$ la medidas de seguridad que se tengan para resguardar el incidente y $D$ un valor asignado a la carga del vehículo. Claramente un robo puede tener otro tipo de repercusiones económicas, como costos de reparación y daños en la vía de transporte. O aún peor perdidas que no pueden ser cuantificar como traumas o perdida de vidas humanas. 

Los factores expuestos en la formula (1) dependen del contexto donde se aplique el problema. Para el estudio técnicas usaremos una definición alternativa que es proporcional al riesgo establecido:

\begin{equation}
    R^r_{i} = \sum_{(i,j)\in r} c_{ij} \cdot D^r_i
\end{equation}

donde $c$ sera la distancia entre los arcos y $D$ la carga del vehículo.


Para mayor detalle sobre las definiciones y un modelo de programación lineal se le atribuye la formulación original (Talarico, Sörensen, Springael) \cite{TALARICO2015457}.




\subsection{Propuestas meta-heurísticas}

La resolución de problemas \textit{VRP} mayoritariamente consiste en formar soluciones de buena calidad a través de distintas heurísticas. Luego se realiza un proceso de búsqueda local para mejorar las soluciones formadas \cite{BRAEKERS2016300}.

Los operadores locales se basan en permutar nodos dentro de la misma ruta o entre distintas. Ejemplo de algunos movimientos son: \textit{Two-opt} (cortar un arco e intercambiar dos subsequencias de nodos dentro o entre dos rutas distintas), \textit{Relocate} (mover un nodo de una ruta a otra), \textit{Exchange} (intercambiar dos nodos entre rutas)  y \textit{Cross-exhange} (intercambiar dos sequencias de nodos consecutivos entre rutas)\cite{braysy2005vehicle}. Una librería más extensa de movimientos es describida por Groër \cite{groer2010library}.


Talarico en 2015 \cite{TALARICO2015457} propone cuatro meta-heurísticas conjunto al planteamiento del problema. Una modificación de la heurística de Clarke y Wright \cite{clarke&wright} para respetar las restricciones de riesgo, esta heurística se basa en hacer rutas exclusivas para cada cliente e ir combinando rutas que reduzcan la función objetivo. Una heurística  basada en los vecinos más cercano conjunto a una selección codiciosa aleatorizada. Otra heurística es una versión modificada usada para encontrar soluciones de \textit{TSP: Traveling Salesman Problem} de manera codiciosa \cite{prins}, se generan soluciones iniciales ignorando las restricciones teniendo un tour completo, luego se reparan dividiendo las rutas conforme a una función objetivo y se detiene cuando sean factibles. Por último una versión determinista para encontrar soluciones basadas en \textit{TSP Lin–Kerninghan} \cite{tlk} con modificaciones \cite{HELSGAUN2000106}\cite{helsgaun2006effective}, la solución obtenida se va dividiendo como se menciono anteriormente. Para escapar los óptimos locales se usa restart y una perturbación que modifica la solución aleatoriamente hasta encontrar una ruta más larga por un factor $\triangle$, $(1+\triangle)\cdot f(x)$.

Una meta-heurística llamada ACO-LNS en 2017 \cite{TALARICO2017547}, encuentra mejores resultados. Esta crea soluciones iniciales de \textit{TSP} usando una técnica de optimización de colonia de hormigas \cite{585892}, luego se dividen \cite{prins} en soluciones factibles para comenzar una búsqueda local. Esta búsqueda se realiza usando \textit{Variable Neighbourhood Descent (VND)} en conjunto de siete movimientos \cite{braysy2005vehicle}, estos fueron elegidos luego de probar una serie de combinaciones para mejorar los resultados. Además se hace uso de revertir las rutas para cambiar el riesgo.

En 2018 un autor hace uso de \textit{Greedy Randomized Adaptive Search Procedure (GRASP)} \cite{grasp1} \cite{grasp2} para solucionar el problema enfocándose en la eficiencia computacional \cite{RADOJICIC2018486}. Este método hace uso del algoritmo de Clarke y Wrights \cite{clarke&wright} para formar soluciones, con una modificación de la función objetivo (\textit{Fuzzy Aproach}). Se hace uso de las restricciones de riesgos para evaluar una ruta, basándose en que las rutas con menores riesgos son de mayor calidad, debido a que se pueden combinar con más sub-rutas. Luego se hace una búsqueda local usando \textit{Path-Relinking} el cuál ha mostrado mejoras en algortimos de búsqueda local para problemas de \textit{VRP} \cite{ho2006path}. Con esta técnica se propone además una estructura de datos eficiente para evaluar soluciones. Básicamente, se almacenan los datos de distancia, dinero y riesgo en ambas direcciones de una ruta en un punto.

En el mismo año se publicó una formulación del problema considerando vehículos con distintas capacidades (\textit{Cargo Theft Weighted Vehicle Routing Problem: CTWVRP}) \cite{Repolho2019}. En el artículo se realiza un caso de estudio de una farmacéutica en Brazil, con distribuciones de probabilidad para distintos sectores de la ciudad de Rio de Janeiro. La técnica utilizada se basa en \textit{Simulated Annealing: SA}, la solución inicial se construye bajo una heurística de distribución asignando aleatoriamente nodos a una cantidad determinada de vehículos. Los vecinos se definen bajo tres movimientos: \textit{Exchange} (permutar el orden de nodos en una ruta), \textit{ShiftEnd} (mueve un cliente de una ruta al final de otra ruta distinta) y \textit{Swap} (intercambiar dos nodos entre rutas distintas). Se realizan distintas pruebas variando los niveles de riesgo permitidos, temperaturas y tiempo de ejecución del \textit{SA} comparando las compensaciones de estas en la calidad de la solución. 


\section{Técnica propuesta}


\subsection{Algoritmo Greedy}
Se usara un algoritmo codicioso determinista para encontrar una solución por bajo coste computacional. El algoritmo parte con una ruta vacía y va visitando clientes conforme le sea posible al vehículo devolverse sin romper la restricción de riesgo. Cuando no pueda visitar otro cliente, se retorna al deposito y se inicia una nueva ruta. 

\begin{algorithm}
\begin{algorithmic}[1]
\Procedure{greedyRCTVRP}{\textit{demands array}, \textit{distance matrix}, $V$, $N$, $R$}
\State let $G$ be an empty list of routes  
\While{there's demand}  
    \State let $r_i$ be route asociated to vehicle $i$.
    \While{$r_i$ not in risk}  
        \State $v$ = choose node($r_i$ , \textit{demands array}, \textit{distance matrix}, $N$, $R$)
        \If {$v$ is chooseable}
        \State mark $v$ as chosen.
        \Else 
        \State $r_i$ is in risk.
        \EndIf
    \EndWhile
    \State update demand.
    \State add $r_i$ to $G$.
\EndWhile
\EndProcedure
\end{algorithmic}
\end{algorithm}


El criterio para escoger un cliente a visitar es una elección codiciosa más una heurística guiar la formación de rutas. Esto es para todos los nodos disponibles, ponderar su distancia entre el nodo y el cliente actual y la distancia al deposito. La ponderación varia mientras se forman se avanza por un factor $\alpha$, el cual incrementa conforme aumenta el riesgo de la ruta.

\begin{equation}
    \text{FO}^s_{ij} = (1-\alpha_i)D_{ij} + \alpha_i D_{je} 
\end{equation}

El factor $\alpha$ sera el ratio entre el riesgo en un punto $i$ y el límite de riesgo del problema:

\begin{equation}
    \alpha_i = \frac{R^r_{i}}{T} 
\end{equation}

De esta manera se espera que el vehículo prefiera nodos más cercanos al deposito al final de su recorrido.

\subsection{Simulated Annealing (SA)}

El segundo algoritmo propuesto se basa en SA. Se parte con una solución aleatoria factible y a lo largo de la ejecución se explora el espacio con cuatro posibles movimientos:


\begin{itemize}
    \item Intra-Swap: Se cambia el orden de dos nodos dentro de una ruta.
    \item Inter-Reverse: Se invierte el orden de una ruta.
    \item Intra-Swap: Se escogen dos rutas y se intercambian un nodo.
    \item Intra-Shift: Se mueve un nodo de una ruta a otra.
    \item Inter-Merge: Se combinan dos rutas.
\end{itemize}

La manera de escoger las rutas y nodos a intercambiar en todos los casos es aleatoria. 

La solución aleatoria inicial se calcula escogiendo un orden aleatorio completo para recorrer todos los clientes. Luego se corta secuencialmente de tal forma que se respete la restricción de riesgo en las subrutas.



\section{Escenario Experimental}

El algoritmo propuesto fue implementado en \textit{C++} en un computador con Intel i3-7100U (4) @ 2.4GHz con 8GB RAM en  Windows 10.

Las instancias usadas son las siguientes:

\begin{itemize}
    \item SET R contiene 180 instancias generadas aleatoriamente, su rango es de 4 a 20 nodos. Cada instancia tiene sus propios niveles de riesgo y distribución de demanda \cite{TALARICO2015457}.
    \item SET V consiste de 70 problemas VRP incorporando la constante de riesgo, este va desde 22 hasta 301 nodos. Cuenta con tres niveles de riesgo por cantidad de nodo \cite{TALARICO2015457}.
    \item SET O y SET S cada uno con 32 instancias, son construidas de tal manera que se conoce la solución óptima de estas. El SET O son figuras con formas de obeliscos y el SET S son figuras espirales, ambas con el deposito en el centro de las figuras \cite{TALARICO2017547}.
\end{itemize}

\section{Resultados obtenidos}

En los estudios que se usaron estos datasets, se propusieron métodos de búsqueda local y heurísticas para generar soluciones iniciales. Para los conjuntos R y V no se conocen las soluciones óptimas.


\section{Conclusión}

En este artículo se presentó un algoritmo codicioso constructor, basado en una heurística que hace uso del riesgo de la ruta para resolver el problema de diseño de rutas para materiales valiosos (RCTVRP). Esta técnica es determinista.

Para trabajos futuros se pueden considerar otras variantes de \textit{VRP} donde el factor de riesgo sea importante. Extendiendo su utilidad para casos de estudio en la industria. Para poder así medir el desempeño y el comportamiento de las meta-heurísticas propuestas en instancias que se asemejen mejor a contextos donde es relevante el riesgo.


% An example of a floating figure using the graphicx package.
% Note that \label must occur AFTER (or within) \caption.
% For figures, \caption should occur after the \includegraphics.
% Note that IEEEtran v1.7 and later has special internal code that
% is designed to preserve the operation of \label within \caption
% even when the captionsoff option is in effect. However, because
% of issues like this, it may be the safest practice to put all your
% \label just after \caption rather than within \caption{}.
%
% Reminder: the "draftcls" or "draftclsnofoot", not "draft", class
% option should be used if it is desired that the figures are to be
% displayed while in draft mode.
%
%\begin{figure}[!t]
%\centering
%\includegraphics[width=2.5in]{myfigure}
% where an .eps filename suffix will be assumed under latex, 
% and a .pdf suffix will be assumed for pdflatex; or what has been declared
% via \DeclareGraphicsExtensions.
%\caption{Simulation Results.}
%\label{fig_sim}
%\end{figure}

% Note that IEEE typically puts floats only at the top, even when this
% results in a large percentage of a column being occupied by floats.


% An example of a double column floating figure using two subfigures.
% (The subfig.sty package must be loaded for this to work.)
% The subfigure \label commands are set within each subfloat command,
% and the \label for the overall figure must come after \caption.
% \hfil is used as a separator to get equal spacing.
% Watch out that the combined width of all the subfigures on a 
% line do not exceed the text width or a line break will occur.
%
%\begin{figure*}[!t]
%\centering
%\subfloat[Case I]{\includegraphics[width=2.5in]{box}%
%\label{fig_first_case}}
%\hfil
%\subfloat[Case II]{\includegraphics[width=2.5in]{box}%
%\label{fig_second_case}}
%\caption{Simulation results.}
%\label{fig_sim}
%\end{figure*}
%
% Note that often IEEE papers with subfigures do not employ subfigure
% captions (using the optional argument to \subfloat[]), but instead will
% reference/describe all of them (a), (b), etc., within the main caption.


% An example of a floating table. Note that, for IEEE style tables, the 
% \caption command should come BEFORE the table. Table text will default to
% \footnotesize as IEEE normally uses this smaller font for tables.
% The \label must come after \caption as always.
%
%\begin{table}[!t]
%% increase table row spacing, adjust to taste
%\renewcommand{\arraystretch}{1.3}
% if using array.sty, it might be a good idea to tweak the value of
% \extrarowheight as needed to properly center the text within the cells
%\caption{An Example of a Table}
%\label{table_example}
%\centering
%% Some packages, such as MDW tools, offer better commands for making tables
%% than the plain LaTeX2e tabular which is used here.
%\begin{tabular}{|c||c|}
%\hline
%One & Two\\
%\hline
%Three & Four\\
%\hline
%\end{tabular}
%\end{table}


% Note that IEEE does not put floats in the very first column - or typically
% anywhere on the first page for that matter. Also, in-text middle ("here")
% positioning is not used. Most IEEE journals/conferences use top floats
% exclusively. Note that, LaTeX2e, unlike IEEE journals/conferences, places
% footnotes above bottom floats. This can be corrected via the \fnbelowfloat
% command of the stfloats package.




% conference papers do not normally have an appendix


% use section* for acknowledgement
%\section*{Acknowledgment}


%The authors would like to thank...





% trigger a \newpage just before the given reference
% number - used to balance the columns on the last page
% adjust value as needed - may need to be readjusted if
% the document is modified later
%\IEEEtriggeratref{8}
% The "triggered" command can be changed if desired:
%\IEEEtriggercmd{\enlargethispage{-5in}}

% references section

% can use a bibliography generated by BibTeX as a .bbl file
% BibTeX documentation can be easily obtained at:
% http://www.ctan.org/tex-archive/biblio/bibtex/contrib/doc/
% The IEEEtran BibTeX style support page is at:
% http://www.michaelshell.org/tex/ieeetran/bibtex/
%\bibliographystyle{IEEEtran}
% argument is your BibTeX string definitions and bibliography database(s)
%\bibliography{IEEEabrv,../bib/paper}
%
% <OR> manually copy in the resultant .bbl file
% set second argument of \begin to the number of references
% (used to reserve space for the reference number labels box)

\bibliographystyle{unsrt}
\bibliography{Referencias}




% that's all folks
\end{document}
